\documentclass[landscape, twocolumn, 9pt]{jsarticle}
\AtBeginDvi{\special{landscape}} 

%余白を指定
\usepackage[margin=1cm]{geometry}

%コマンドを再定義
\makeatletter
	%sectionのフォントサイズをnormalに
	\renewcommand{\section}{%
	\@startsection{section}{1}{\z@}%
	{1.5\Cvs \@plus.5\Cdp \@minus.2\Cdp}%
	{.5\Cvs \@plus.3\Cdp}%
	{\reset@font\normalsize\bfseries}}
	%目次表題を変更
	\renewcommand{\contentsname}{\large ProconLibrary}
\makeatother

%フッター, ヘッダー
\usepackage{fancyhdr}
\pagestyle{fancy}
\lhead{\leftmark}
\rhead{\thepage}
\lfoot{ichyo}
\rfoot{last modified: \today}

%ソースコードの表示
\usepackage{listings, jlisting}
\usepackage{ascmac,here,txfonts,txfonts}
\usepackage{color}
\lstset{language=c++,
        numbers=left,
        breaklines=true,
        frame=single,
        basicstyle={\small\ttfamily},
        commentstyle={\color[cmyk]{1,0.4,1,0}},
        keywordstyle={\bfseries \color[cmyk]{0.4,1,1,0}},
        stringstyle={\ttfamily \color[rgb]{0,0,0.6}},
        framesep=2pt,
        numbersep=5pt,
        numberstyle=\scriptsize,
        tabsize=2}

\begin{document}

%目次(subsectionまで表示)
\setcounter{tocdepth}{2}
\tableofcontents
\clearpage

% 本文
\section{Common}
\subsection{設定ファイル}
\lstinputlisting[language=]{../common/vimrc}
\lstinputlisting[language=]{../common/xmodmap}
\subsection{テンプレート}
\subsubsection{common.h}
\lstinputlisting{../common/common.h}
\subsubsection{graph.h}
\lstinputlisting{../common/graph.h}
\subsubsection{graph\_weight.h}
\lstinputlisting{../common/graph_weight.h}
\subsubsection{graph\_flow.h}
\lstinputlisting{../common/graph_flow.h}

\section{グラフ}
\subsection{用語集}
\begin{description}
    \item[マッチング] 辺の集合.どの2辺も端点を共有しない.
    \item[辺カバー] 辺の集合.すべての点に対して,それに接続する辺のいずれかが入っている.(辺で点をカバーする)
    \item[安定集合] 点の集合.どの2点も隣接していない.
    \item[点カバー] 点の集合.すべての辺に対して,端点のどちらかが入っている.(点で辺をカバーする)
    \item[DAG] Directed Acyclic Graph.閉路を持たない有向グラフ.
    \item[パス] 頂点の列$v_1, v_2, \dots, v_k$.隣り合う2頂点間に辺が存在する.
    \item[パスカバー] パスの集合.すべての頂点が,集合内のパスによってカバーされる.
    \item[独立パスカバー] パス同士の間に頂点の共有がないようなパスカバー.
    \item[推移性のあるグラフ] 頂点aからbと,bからcに辺があるなら,必ずaからcにも辺があるようなグラフ.
    \item[有向グラフの二部グラフ化] 1つの頂点vを2つの頂点v[入口]とv[出口]に分けて,vからuへの辺があるなら,v[出口]からu[入口]に辺を引く.
\end{description}
\subsection{公式集}
\subsubsection{オイラーの多面体定理}
連結な平面グラフについて,頂点数$V$, 辺数$E$, 面数$F$の関係は外面も含めて$V - E + F = 2$
\subsubsection{マッチングなどの関係式} % 要確認
\begin{itemize}
    \item $|最大マッチング| \leq |最小点カバー|$ (二部グラフでは等号成立) 
    \item $|最大安定集合| + |最小点カバー| = |V|$
    \item 孤立点のないグラフについて,$|最大マッチング| + |最小辺カバー| = |V|$
    \item 二部グラフについて,$|最小辺カバー| = |最大安定集合|$
    \item $X \in V(G)がGの点カバー \iff V(G) - XはGの安定集合$
\end{itemize}
\subsubsection{パスカバーの関係式}
\begin{itemize}
    \item $|最小パスカバー| \leq |最小独立パスカバー| \leq |最大安定集合|$
    \item 推移性のあるグラフについて,$|最小パスカバー| = |最大安定集合|$
    \item DAGについて,$|V| - |二部グラフ化したものの最大マッチング| = |最小独立パスカバー|$
    \item 推移性のあるDAGについて,$|最大安定集合| = |最小独立パスカバー| = |最小パスカバー| = |V| - |二部グラフ化したものの最大マッチング|$
\end{itemize}

\subsection{最短経路}
\lstinputlisting{../graph/shortest_path.cpp}

\subsection{トポロジカルソート}
\lstinputlisting{../graph/topological_sort.cpp}

\subsection{強連結成分分解}
\lstinputlisting{../graph/strongly_connected_component.cpp}

\subsection{2-SAT}
\lstinputlisting{../graph/2sat.cpp}

\subsection{最大二部マッチング(Ford Furkerson) $O(V (V + E))$}
二部グラフの最大マッチングの大きさを求める.
\lstinputlisting{../graph/bipartite_matching.cpp}

\subsection{最大二部マッチング(Hopcroft-Karp) $O(E \sqrt{V})$}
 Ford Furkersonよりも高速に二部グラフの最大マッチングを求めるアルゴリズム
\lstinputlisting{../graph/hopcraft_karp.cpp}

\subsection{最大流(Dinic) $O(EV^2)$}
最大流問題を解くアルゴリズム.実用上高速なアルゴリズム
(ランダムケースなどの得意なグラフだと体感でO(VE)とかO(V + E)くらいの速さで動くらしい)

 注意:\verb|max_flow()|を何度も実行するときはgraphを保存しておくこと.(\verb|max_flow()|でグラフの情報が壊れる)
 \lstinputlisting{../graph/maximum_flow.cpp}

\subsection{最小費用流(Primal-Dual) $O(F E log V)$ ,Fは流量}
始点sから終点tまでの流量fのフローでコストが最小のものを求めるアルゴリズム.

注意:辺のコストが負のときは使えない, 多重辺や自己辺を入れてはいけない
\lstinputlisting{../graph/min_cost_flow.cpp}

\subsection{最近共通祖先}
\lstinputlisting{../graph/lca.cpp}

\subsection{最小シュタイナー木 $O(n4^t + n^2 2^t + n^3)$}
シュタイナー木 := $V$の部分集合$T$に対して,$T$のすべての頂点を含む木のこと
\lstinputlisting{../graph/minimum_steiner_tree.cpp}

\subsection{最小全域有向木 $O(V E)$}
\begin{itemize}
        \item 強連結成分分解が必要
        \item グラフが壊されることに注意
\end{itemize}
\lstinputlisting{../graph/minimum_optimum_branching.cpp}

\subsection{支配集合問題}
\lstinputlisting{../graph/minimum_dominating_set.cpp}

\subsection{極大/最大 独立集合・クリーク}
\lstinputlisting{../graph/maximum_independent_set.cpp}

\subsection{オンライントポロジカルソート}
\lstinputlisting{../graph/online_topological_sort.cpp}

\section{データ構造}
\subsection{UnionFind}
\lstinputlisting{../data_structure/union_find.cpp}

\subsection{BIT}
\lstinputlisting{../data_structure/BIT.cpp}

\subsection{Treap}
\lstinputlisting{../data_structure/treap.cpp}

\subsection{Lower Envelope}
\lstinputlisting{../data_structure/lower_envelope.cpp}

\section{動的計画法}
\subsection{編集距離}
\lstinputlisting{../dynamic_programming/edit_distance.cpp}

\subsection{ナップサック問題(近似)}
\lstinputlisting{../dynamic_programming/knapsack_approximation.cpp}

\subsection{最長共通部分列}
\lstinputlisting{../dynamic_programming/longest_common_subsequence.cpp}

\section{文字列}
\subsection{Trie}
\lstinputlisting{../string/trie.cpp}
\subsection{Aho corasick}
\lstinputlisting{../string/aho_corasick.cpp}
\subsection{Suffix Array}
\lstinputlisting{../string/suffix_array.cpp}
\subsection{Rolling Hash}
\lstinputlisting{../string/rolling_hash.cpp}

\section{数学}
\subsection{組み合わせ数}
\lstinputlisting{../math/combination.cpp}

\subsection{二次方程式・三次方程式}
\lstinputlisting{../math/equation.cpp}

\subsection{ユークリッドの互除法}
\lstinputlisting{../math/euclid.cpp}

\subsection{ガウスジョルダン}
\lstinputlisting{../math/gauss_jordan_elimination.cpp}

\subsection{Givens Elimination}
\lstinputlisting{../math/givens_elimination.cpp}

\subsection{行列演算}
\lstinputlisting{../math/matrix.cpp}

\subsection{miller rabin素数判定}
\lstinputlisting{../math/miller_rabin.cpp}

\subsection{剰余演算}
\lstinputlisting{../math/mod.cpp}

\subsection{剰余クラス}
\lstinputlisting{../math/mod_class.cpp}

\subsection{二分探索・三分探索}
\lstinputlisting{../math/optimization.cpp}

\subsection{有理数クラス}
\lstinputlisting{../math/rational.cpp}

\subsection{区間篩}
\lstinputlisting{../math/segment_sieve.cpp}

\subsection{数値積分}
\lstinputlisting{../math/simpson.cpp}

\subsection{Z変換}
\lstinputlisting{../math/z_transform.cpp}

\section{幾何}
\subsection{公式集}
\begin{description}
    \item[正弦定理]
        外接円の半径をRとすると,\[a = 2R\sin A\]
    \item[余弦定理]
        \[a^2 = b^2 + c^2 - 2bc\cos A\]
        \[\cos A = \frac{b^2 + c^2 - a^2}{2bc}\]
    \item[ヘロンの公式]
        $s = \frac{a + b + c}{2}$とおくと面積$S$は,
        \[S = \sqrt{s(s - a)(s - b)(s - c)}\]
    \item[三角形の面積(1辺と両端角)]
        \[S = \frac{a^2\sin B \sin C}{ 2\sin(B + C) }\]
\end{description}
\subsection{Point}
\lstinputlisting{../geometry/point.cpp}
\subsection{Line}
\lstinputlisting{../geometry/line.cpp}
\subsection{Polygon}
\lstinputlisting{../geometry/polygon.cpp}
\subsection{Circle}
\lstinputlisting{../geometry/circle.cpp}
\lstinputlisting{../geometry/circle_tangent.cpp}
\subsection{Convex}
\lstinputlisting{../geometry/convex.cpp}
\subsection{3次元幾何}
\lstinputlisting{../geometry/3dgeometory.cpp}
\subsection{Triangle}
\lstinputlisting{../geometry/triangle.cpp}
\subsection{線分アレンジメント}
 複数の線分が与えられたとき,次のグラフを作成する.
 \begin{itemize}
   \item 線分の端点と線分対の交点を頂点として持ち,
   \item 二頂点が同一線分に乗っていて二頂点の間に別の頂点は無いときに二頂点間距離を重みとする辺を持つ
 \end{itemize}
 ここではグラフのサイズを減らすために追加条件を課している.この条件は推移的な辺を省略したことになる.
 省略したくない場合はグラフ構成のときに cut のすべての対について辺を張ればよい.

 注意.入力の線分の集合に オーバーラップする線分は無いものとする.オーバーラップする線分がある可能性がある場合は,あらかじめ merge すること.

\lstinputlisting{../geometry/segment_arrangement.cpp}

\section{その他}
\subsection{bit演算}
\lstinputlisting{../others/bit.cpp}
\subsection{日付計算}
\lstinputlisting{../others/date.cpp}
\subsection{ダイス}
\lstinputlisting{../others/dice.cpp}

\end{document}
